\documentclass{article}

\usepackage[activate={true,nocompatibility},final,tracking=true,kerning=true,spacing=true,factor=1100,stretch=10,shrink=10]{microtype}
% activate={true,nocompatibility} - activate protrusion and expansion
% final - enable microtype; use "draft" to disable
% tracking=true, kerning=true, spacing=true - activate these techniques
% factor=1100 - add 10% to the protrusion amount (default is 1000)
% stretch=10, shrink=10 - reduce stretchability/shrinkability (default is 20/20)

\usepackage[utf8]{inputenc}
\usepackage{amssymb}
\usepackage{amsmath}
\usepackage{mathtools}
\usepackage{svg}
\usepackage{graphicx}
\usepackage[superscript, biblabel, nomove]{cite}
\usepackage[hidelinks]{hyperref}
\usepackage{cleveref}
\usepackage{tikz}
\usepackage{pgfplots}
\usepackage{datetime}
\usepackage{color}

% Use § for sections
\crefformat{section}{\S#2#1#3}
\crefformat{subsection}{\S#2#1#3}
\crefformat{subsubsection}{\S#2#1#3}

\usepackage{appendix}

% == Remove or comment these out for release
%\usepackage{todonotes}
%\usepackage{draftwatermark}
%\SetWatermarkText{DRAFT}
%\SetWatermarkScale{3}
% ==

\usepackage[utf8]{inputenc}
\usepackage[english]{babel}
\usepackage{amsthm}

% Make itemize bullets slightly smaller
\renewcommand\labelitemi{{\boldmath$\cdot$}}

\newtheorem{theorem}{Theorem}
\theoremstyle{definition}
\newtheorem{definition}{Definition}[section]

\newtheorem{thm}{The`orem}[section] % the main one
\newtheorem{lemma}[thm]{Lemma}

\theoremstyle{plain} % just in case the style had changed
\newcommand{\thistheoremname}{}
\newtheorem{genericthm}[thm]{\thistheoremname}
\newenvironment{namedthm}[1]
  {\renewcommand{\thistheoremname}{#1}%
   \begin{genericthm}}
  {\end{genericthm}}

\begin{document}

% Macros
\newcommand{\HAV}{\textsc{hav}}
\newcommand{\NOM}{\textsc{nom}}


\title{Havven: a stablecoin system\\ v0.6}
\author{Samuel Brooks, Anton Jurisevic, Michael Spain, Kain Warwick}
\date{}

\begin{figure}
    \centering
    %\includesvg[width=0.33\textwidth]{img/block8logo}
    \includegraphics[width=0.33\textwidth]{img/havvenlogo}
\end{figure}
\maketitle

\hfill

\begin{abstract}
\noindent An effective decentralised payment network must not rely on a central authority to maintain trust, and must utilise a stable medium of exchange. Attempts to create digital currencies prior to Bitcoin were centralised, making them vulnerable to censorship and seizure. Bitcoin’s distributed consensus mechanism protected it from interference, but its fixed monetary policy fostered extreme volatility. Havven solves these problems by issuing a price-stabilised token against a distributed collateral pool which derives its value from the utility of the system. Fees are levied on transactions, and they are dispersed proportionally among collateral token holders. Growth in transaction volume thus increases the value of the collateral, which allows the stable token supply to expand to meet demand. The resulting token retains the best features of Bitcoin, as it eliminates centralisation and maintains censorship resistance. The introduction of price stability enables the resulting payment network to be used for everyday economic purposes, thereby accelerating the adoption of decentralised systems.
\end{abstract}	
\vspace{20mm}
\begin{center}
  %\color{gray}
  \small{\textit{\currenttime , \today}}
\end{center}

% \todo[inline]{Incorporate white paper criticisms.}

\pagebreak 

\tableofcontents

% == Remove or comment out this section for release
% \pagebreak
% \section{TODO}
% \listoftodos{}
% ==

\pagebreak

\section{Introduction}

\subsection{Money and Cryptocurrencies}

\noindent The technology of money has three key functions: to act as a unit of account, a medium of
exchange and a store of value. In addition, money should ideally exhibit durability,
portability, divisibility, uniformity, limited supply, and acceptability.
As payment technology has advanced in recent years, it has become increasingly invisible
and it is often lost upon users of money that, like any technology, it can be improved.
Specifically, this means improving the performance of those desirable properties. \\

\noindent Bitcoin is an impressive technological advancement on existing forms of money because it simultaneously improves durability, portability, and divisibility. Further, it does so without requiring centralised control or the enforcement of a nation state from which to derive its value. This fixed monetary policy has protected Bitcoin from debasement and devaluation, allowing it to outperform other forms of money as a store of value. But this has created the potential for short-run volatility as Bitcoin lacks a mechanism to dynamically adjust to changing demand. \\

\noindent Bitcoin has thus tended to be a poor medium of exchange and an even worse unit of account.
In order for something to perform these functions it must remain
relatively stable against the price of goods and services over the short to medium term.

\subsection{Stablecoins}

\noindent A stablecoin is a cryptocurrency designed for price stability, such that it can function both as a medium of exchange and unit of account. It should ideally be as effective for making payments
as fiat currencies like the US Dollar, but still retain the desirable characteristics of Bitcoin, namely
transaction immutability, censorship resistance and decentralisation. \\

\noindent Cryptocurrencies that exhibit these characteristics are clearly a far better form of money; but adoption has been hindered by the volatility inherent in the inflexible monetary policies employed by decentralised systems. Thus stability continues to be one of the most valuable yet elusive characteristics for the technology. Clearly, efforts to create dynamic monetary policies within cryptoeconomic systems is still nascent, and significant research into stable monetary frameworks for cryptocurrencies is required. \\

\noindent The interested reader can also find additional discussion of stablecoins,
cryptoeconomics, competitors, and other related topics on our blog at \href{http://blog.havven.io}{\texttt{http://blog.havven.io}}.

\subsection{Havven}

\noindent Havven is a decentralised stability marketplace, where users requiring stability transact directly with those who provide the collateral to create it. This enables a novel form of representative money in which there is no requirement for a physical asset, thus removing problems of trust and custodianship. The asset used to back the stablecoin is a pool of reserve tokens that collectively represent the system itself; controlling these reserve tokens reflects participation in the Havven system, and they are a proxy for its value. Havven generates fees from users who transact in the stablecoin and distributes them among the holders of the reserve token, compensating them for underpinning the system. Havven therefore rewards those who actively participate in maintaining the stability of the system and charges those who benefit from its utility. These rewards are proportionally applied in response to the active management of the supply of the exchange token such that its price mirrors that of the asset it tracks. \\

\noindent Because we have created a system that generates cash flow for participants, we now have an asset which can be used as the collateral to support the stablecoin with a well-defined market value. The key to this is that the value of the system is measured in USD. This allows the system to issue a stablecoin which can be presented and redeemed for a percentage of the collateral tokens valued at 1 USD. Backing a stablecoin in this way is beneficial because such a cryptoeconomic system does not require trust in a centralised party; each participant has full transparency over how many tokens have been issued against the available collateral at all times. \\

\noindent The two linked tokens and the complex of incentives are described below: \\

\noindent \textbf{Havven}

\vspace{1mm}

\noindent The collateral token, whose supply is static.
The capitalisation of the havvens in the market reflects both the system's aggregate value and the reserve
which backs the stablecoin. Thus, users who hold havvens take on the role of maintaining stability.
Following Bitcoin, the Havven system will appear in upper case and singular; while the havven
token will be lower case and may be plural.

\vspace{2mm}

\noindent \textbf{Nomin}

\vspace{1mm}

\noindent The exchange token - the stablecoin - whose supply floats.
Its price as measured in fiat currency should be relatively stable.
Other than price stability, the system should also encourage some adequate level
of liquidity for nomins to act as a useful medium of exchange. \\

\noindent Each holder of havvens is able to issue a value of nomins in proportion to the USD value
of the havvens they hold and are willing to place into escrow. If the user wishes to release their escrowed havvens, they must present the system with nomins in order to free their havvens and trade them again.
The holders of this token provide both collateral and liquidity, and in so doing assume some
level of risk. To compensate this risk, such nomin-issuers will be rewarded with fees the system levies
automatically as part of its normal operation. \\

\subsection{Design Rationale}

\noindent This issuance mechanism allows nomins to act as a form of representative money, where each nomin represents a share in the havven value held in reserve. Nomins derive value insofar as they provide a superior medium of exchange, and are effectively redeemable for a constant value of the denominating asset. In this paper, we use USD as this asset, but this could be any external and appropriately fungible asset, such as a commodity or a fiat currency. The system incentivises the issuance and destruction of nomins in response to changes in demand, to maintain a constant price. \\

%\noindent The design choice to back the system with a self-referential token was obvious;
%an asset-backed stablecoin with a cryptocurrency basket as reserve will always be inherently
%volatile, despite diversification, and will never be able to achieve the bespoke functionality
%of an asset which derives its value from stability. \\

\noindent In order to issue a nomin, there must be a greater value of havvens escrowed in the system. Overcollateralisation in this way provides confidence that nomins can be redeemed for their face value, even if there is a reduction in the value of the locked havvens. For example, if the collateralisation ratio is set to 100:1 then the price of havvens would need to fall 99\% for the nomin to no longer be overcollateralised. On the other hand, the ratio could be set to 1:1, but any decrease in havven price would mean the currency was no longer overcollateralised. In this situation, it would not be possible for all nomins to be redeemed for their face value, which may cause confidence in the system to drop.  \\

\noindent While overcollateralisation strongly supports a stable nomin price, it simultaneously reduces the efficiency of the system by limiting the utility of the nomin. A ratio of 1:1 provides far greater utility than 100:1, with respect to both the effiicency of the collateral and the revenue generated from transaction fees. There is clearly a trade-off between efficiency and resilience in Havven, which means the collateralisation incentives must therefore not only promote nomin price stability. They must also allow the system to operate as efficiently as possible, whilst maintaining resilience to reduction in the value of the collateral. The design of these incentives is described in detail in section 2. \\

\noindent Due to denominating the value of the stablecoin in an external fiat currency stability is relative only to that currency. In the future the system may support additional 'flavors' of stablecoin that are denominated in other currencies such as the Euro. Denominating the stablecoin in a fiat currency removes the need to respond to macroeconomic conditions, as it benefits from the stabilisation efforts of large institutions acting in fiat markets. \\

\pagebreak

%\input{tex/design_considerations}
\section{System Description} Havven is designed to incentivise stability in a decentralised cryptocurrency denominated in some external currency, such as the USD. The dual-token system is combined with a set of novel incentive mechanisms designed to stabilise the price of one of the two tokens. The first token is the havven token, \HAV{} - to avoid confusion with the system itself - and the nomin, \NOM{} (short for denominator). \\

\noindent \HAV{} serves two functions:

\begin{enumerate}
\item{To provide the system with collateral (the system itself is tokenised).}
\item{To allow actors to contribute to the price stabilisation process through a set of incentives.}
\end{enumerate}

\noindent The second token, \NOM{}, is the stablecoin. The purpose of \NOM{} is to track the price of a chosen external denominating currency via the actions of \HAV{} token holders; holders of \HAV{} participate in modifying the level of supply in the \NOM{} market such that the market price of \NOM{} is maximally stable.

\subsection{Incentive Layering}

We classify the various incentives that can be applied in a stablecoin system. Note that any subset of these can be linearly combined in order to produce a sophisticated and powerful incentive structure. Havven's approach to achieving price stability is to be as passive as possible and only switch on higher levels of incentivisation when necessary. The order in which these categories appear is the order in which they are applied:

\paragraph{Overcollateralisation}

The basis for price stability within Havven is overcollateralisation of stablecoin value. This means the value of the escrowed collateral backing the stablecoin is strictly greater than the value of \NOM{} in circulation. In Havven, this ratio of \NOM{} to \HAV{} is known as the utilisation ratio.

\paragraph{Fees}

\noindent The second layer of economic incentives for \HAV{} holders is to provide them with fees in accordance with their performance in adjusting the supply of \NOM{}. These fees are generated from small charges on all \NOM{} transfers. The fees are directed to the \HAV{} holders as a reward for helping maintain the correct supply of \NOM{}.

\paragraph{Interest Rates}

\noindent Interest rates on \HAV{} can be applied in addition to the application of fees, in either fixed or floating \HAV{} supply regimes. Interest rates will be discussed in a future iteration of the whitepaper.

\paragraph{Collateral Recovery}

\noindent As a final layer of incentives, forced recovery of an actor's escrowed \HAV{} may be required in order to equilibriate individual positions of utilisation ratio. Collateral recovery will be discussed in a future iteration of the whitepaper.

\subsection{Overcollateralisation}

\noindent We first introduce the core system variables:

\begin{align*}
H &= \text{Quantity of \HAV{},} & N &= \text{Quantity of \NOM{},} \\
P_h &= \text{\HAV{} Price,}  & P_n &= \text{\NOM{} Price.}
\end{align*}

\noindent All \HAV{} tokens are created in the initial system state, so $H$ is constant. The quantity of \NOM{}, $N$, floats in response to the actions of \HAV{} holders, who, for the most part, are assumed to act in accordance with their incentives, thereby encouraging the \NOM{} price, $P_n$, to stabilise with changes in demand.

\subsubsection{Nomin Supply Control}

\noindent \NOM{} can only be issued when a \HAV{} holder decides to escrow some number of \HAV{} under their control. Once the \HAV{} have been escrowed (via smart contract) a quantity of \NOM{} are generated equal in value to the value of escrowed \HAV{} multiplied by the maximum utilisation ratio. This ensures the value of the \NOM{} that is produced is less than the value of the backing \HAV{} collateral. \\

\noindent The system then immediately places a \textbf{limit sell} order with a price of \$1 on an exchange (such as a dedicated decentralised exchange for \HAV{} and \NOM{} trading). This means that the \NOM{} will be sold at the current market price, down to a minimum price of \$1 USD. If we assume implementation on Ethereum, then the \NOM{} are sold for an amount of ETH valued at \$1, with the proceeds of the sale remitted to the issuer. \\

\noindent It is important for the proper functioning of the system that the pool of \NOM{} is always overcollateralized by the value of \HAV{}. The \textbf{utilisation ratio} is what initialises this property.

\subsubsection{Utilisation Ratio}

\noindent The utilisation ratio is defined by the total value of \NOM{} against the total value of \HAV{}:

$$ U = \frac{P_n * N}{P_h * H} $$ \\

\noindent Intuitively, if $U = 1$, the value of \NOM{} and \HAV{} are equal. Hence, given our overcollateralisation property, our target $U <  1$. To do this, Havven only allows the issuance of \NOM{} up to a maximum utilisation ratio.

\subsubsection{Collateralisation Target}

\noindent Given the need to adjust the supply of \NOM{}, a target utilisation ratio is defined as the point at which maximum incentives are applied:

$$ 0 \leq U \leq U_{max} \leq 1.$$

\noindent Because individual \HAV{} holders have a unique utilisation ratio, $ U_i $, the system can measure the degree to which their $ U_i $ is above or below the target and adjust their incentives accordingly. In this way the system incentivises the creation and destruction of \NOM{}. $ U_{target} $ is defined formally below in terms of $ P_n $ (as \NOM{} price diverges from the desired \$1, increasing incentives are applied to either expand or contract the supply).

\subsubsection{Releasing Havvens from Escrow}

\noindent In order to access the original \HAV{} that have been escrowed, the owner must return the same quantity of issued \NOM{} to the system for destruction. This is known as 'burning' the \NOM{}.

\newpage
\subsection{Nomin Demand and Supply} 

\noindent Demand and supply economics shows that there exists some optimal supply of \NOM{} where the related level of demand yields an equilibrium price of \$1. We can express this quantity in terms of an optimum utilisation ratio, $U_{opt}$. The graph below visualises this situation. \\

\begin{tikzpicture}[scale=3]

% draw axes
\draw [<->, thick] (0,2) node (yaxis) [above] {$P$} |- (2.5,0) node (xaxis) [right] {$Q$};

% draw intersecting lines
\draw (0.5, 0.5) coordinate (a_1) -- (2,1.8) coordinate (a_2);
\draw (0.5, 1.8) coordinate (b_1) -- (2,0.5) coordinate (b_2);

% calculate coordinate of intersection
\coordinate (c) at (intersection of a_1--a_2 and b_1--b_2);

\draw[dashed] (yaxis |- c) node [left] {$1$} -| (xaxis -| c) node[below] {$u_{opt}$};

\end{tikzpicture}
	
\subsubsection*{Demand}

\noindent The system is unable to influence the demand for \NOM{}. We assume that some level of demand exists given the utility of \NOM{} as a stable cryptocurrency.

\subsubsection*{Supply}

\noindent However where demand is unable to be directly influenced, the supply of \NOM{} is controlled by \HAV{} holders who use the system to issue and burn \NOM{} in response to its incentives. Maximum incentive is achieved when $U_i = U_{opt}$, such that $P_n = 1$. $U_{opt}$ will be discussed further below. \\

\newpage
\subsection{Fees} Every time a \NOM{} transaction occurs, the Havven system charges a small transaction fee. Transaction fees allow the system to generate revenue, which it can distribute to \HAV{} holders as an incentive to maintain \NOM{} supply at $U_{opt}$.

\subsubsection{Transaction fees}

\noindent The fee rate charged on \NOM{} transactions is $\alpha_c$. It is constant and will be sufficiently small that it provides little to no friction for the user.\\

$$ a_c = k.$$ 

\begin{tikzpicture}[scale=3]

% draw axes
\draw [<->, thick] (0,2) node (yaxis) [above] {$\alpha_c$} |- (2.5,0) node (xaxis) [right] {$P_N$};

% draw one line
\draw (0.0, 0.75) coordinate (a_1) -- (2.0,0.75) coordinate (a_2);

\coordinate (c) at (1.75, 0.75);

\draw[dashed] (yaxis |- c) node [left] {k};

\end{tikzpicture}

\newpage
\subsubsection{Fee distribution}

\noindent The fee rate paid to a \HAV{} holder that has escrowed their \HAV{} is $\alpha_r$. This rate changes with respect the individual's unique utilisation ratio, $U_i$. It increases linearly to a maximum at the optimal utilisation ratio $U_{opt}$, before quickly diminishing as $U_i$ approaches $U_{max}$. Beyond the maximum utilisation ratio $\alpha_r$ is 0. Note, $\alpha_r$ is applied to the pool of collected fees which is determined by $\alpha_c$. \\

\[
\alpha_r = 
\begin{cases}
 \frac{\alpha_{base}}{U_{opt}} * U_i &\mbox{when } U_i \leq U_{opt}, \\[1em]
 \frac{\alpha_{base}}{U_{max} - U_{opt}} * (U_i  - U_{max}) &\mbox{when } U_{opt} \leq U_i \leq U_{max}, \\[1em]
 0 &\mbox{otherwise}.
 \end{cases}
\]

\begin{tikzpicture}[scale=3]

% draw axes
\draw [<->, thick] (0,2) node (yaxis) [above] {$\alpha_r$} |- (2.5,0) node (xaxis) [right] {$U_i$};

% draw two lines
\draw (0.0, 0.0) coordinate (a_1) -- (1.75,0.75) coordinate (a_2);
\draw (1.75, 0.75) coordinate (a_1) -- (2.25,0.0) coordinate (a_2);

\coordinate (c) at (1.75, 0.75);

\draw[dashed] (yaxis |- c) node [left] {$a_{base}$} -| (xaxis -| c) node[below] {$u_{opt}$};

\end{tikzpicture}

\noindent This fee distribution curve encourages \HAV{} holders who have escrowed to maintain their $U_i$ at $U_{opt}$.  \\

\noindent We have introduced the concept of an optimal utilisation ratio and its importance in achieving $P_n = 1$. However, the system needs a function to determine what $U_{opt}$ is. \\

\noindent If $P_n > 1$ then the system must encourage more \NOM{} to be issued. If $P_n < 1$, the system must encourage \NOM{} to be burned. The definition of $U_{opt}$ must therefore provide this incentive.

\newpage
\subsection{Utilisation Ratio}
\subsubsection{Optimal Utilisation Ratio}

\noindent The optimal utilisation ratio $U_{opt}$ is a target for \HAV{} holders to reach in order to maximise the amount of fees they receive. $U_{opt}$ is defined in terms of $P_n$ such that \HAV{} holders can influence the price of \NOM{} through directly controlling the supply of \NOM{} (a havven holder can change their individual utilisation ratio by buying or issuing more nomins). \\
 
\noindent The function for $U_{opt}$ given below provides our dynamic target for \HAV{} holders based on the price of \NOM{}. The curve shows that the when $P_n$ is close to \$1, $ f'(P_n) $ is small. However, the further $P_n$ diverges from \$1, the larger the derivative becomes, providing an increasing incentive (via fees) for a havven holder to move toward $U_{opt}$.

$$ U_{opt} = f(P_n) * U,$$
$$ f(P_N) = max(\sigma * (x - 1)^{\phi} + 1, 0), $$
$$\text{where } 0 \leq \sigma, \text{ the price sensitivity parameter}, $$
$$\phi \geq 1, \text{ the flattening parameter}. $$ \\

\begin{tikzpicture}[scale=1.15]
\begin{axis}[
    axis lines = left,
    xlabel = $P_N$,
    ylabel = {$f(P_N)$},
    xtick = {1},
    ytick = {0},
    y label style={at={(axis description cs:.15,1.00)}, rotate=270,anchor=south},
    x label style={at={(axis description cs:1.05,.13)},anchor=north},
]
\addplot[
domain=0:2,
range=0:2
]
{(x-1)^3 + 1};
\addplot [
domain=0:2,
range=0:2,
dashed
]  
 {x};
 \addplot[dashed, samples=50, smooth,domain=0:6] coordinates {(1,1)(1,0)};
\end{axis}
\end{tikzpicture}

\newpage

\subsection{Maximum Utilisation Ratio}

\noindent Havven seeks to maintain $U < U_{max} < 1$, in order to remain overcollateralised. It might seem intuitive that $U_{max}$ should be a static value. However, since $U_{opt}$ changes linearly with $P_n$ and inversely with $P_h$, there are several situations where $U_{max}$ may need to change. Below we define $U_{max}$. \\

\[
U_{max} = 
\begin{cases}
 U_{base} &\mbox{when } U_{opt} \leq U_{base}, \\[1em]
 a * U_{opt} &\mbox{otherwise}.
 \end{cases}
\]

\begin{tikzpicture}[scale=3]

% draw axes
\draw [<->, thick] (0,2) node (yaxis) [above] {$U_{max}$} |- (2.5,0) node (xaxis) [right] {$U_{opt}$};

% draw intersecting lines
\draw (0.0, 0.75) coordinate (a_1) -- (0.75,0.75) coordinate (a_2);
\draw (0.75, 0.75) coordinate (a_1) -- (1.8,1.8) coordinate (a_2);

% calculate coordinate of intersection
\coordinate (c) at (0.75,0.75);

\draw[dashed] (yaxis |- c) node [left] {$U_{base}$} -| (xaxis -| c) node[below] {$U_{base}$};

\end{tikzpicture}

\newpage

\subsection{Intrinsic Havven Price} With the \HAV{} token being ERC20 compliant, it will have a market price on both decentralised and centralised exchanges. \\

\noindent While the Havven system will access the current market price via a price oracle, it is beneficial to define a $P_h$ that can be determined internally to avoiding the influence of speculation. Ignoring speculative demand, $P_h$ can be expressed as a function of the transaction fees that the system charges. Below we define an initial iteration of the intrinsic $P_h$.

\begin{align*} 
P_{h,t} &= \frac{1}{H}* \sum\limits_{t=1}^\infty \frac{d_{n,t} *v_{n,t} * \alpha_{R,t}}{(1+R)^t} \approx \frac{d_{n,t} *v_{n,t} * \alpha_{R,t}}{R * H}, \\
& P_{h,t} \text{ is the price of one \HAV{} at time } t, \\
& H \text{ is the number of havvens}, \\
& d_{n,t} \text{ is the demand for \NOM{} at t}, \\
& v_{n,t} \text{ is the velocity of \NOM{} at t}, \\
& \alpha_{R,t} \text{ is the fee from trade with \NOM{}}, \\
& R \text{ is the interest rate / rate of return of havvens}. \\
\end{align*}

\newpage

\subsection{Example Use Case}

\noindent The issuance concept is best understood using an example:
\begin{enumerate}
\item{Bob purchases 10 \HAV{} at \$10 each, total value \$100.}
\item{The maximum utilisation ratio is 0.2.}
\item{Bob decides to escrow all of his \HAV{}, equivalent to 20 \NOM{}. These \HAV{} are now not able to be traded.}
\item{The system sells 20 \NOM{} on the market and transfers the proceeds, in ETH, to Bob's wallet.}
\item{Bob is free to use the ETH in his wallet in any way, including retaining it for the future purchase of \NOM{}.}
\item{In order to release the escrowed \HAV{}, the same number of \NOM{} (20) must be returned to the system, even if they have changed in value to say \$21 or \$19.}
\end{enumerate} 

\noindent Some questions may have already arisen in the reader's mind: \\

\noindent \emph{1. Does Bob have to lock all of his \HAV{} into escrow?} \\ 

\noindent There is no requirement for Bob to escrow all of his \HAV{}; he can escrow as many as he likes. The quantity of \NOM{} that is sold on the market is $ P_h * H_e * U_{max} $ where $H_e$ indicates the quantity of \HAV{} that was escrowed. \\

\noindent \emph{2.What if Bob would like to release his \HAV{}? Where would he acquire \NOM{}?} \\ 

\noindent He simply needs to purchase them in the open market. Assuming an implementation on Ethereum, \HAV{} and \NOM{} would both be ERC20-compatible tokens able to be traded on a variety of centralised and decentralised exchanges. Once Bob buys $20$ \NOM{}, he can present them to the system to be burned, thus releasing the escrowed \HAV{} back to him. \\ 

\noindent \emph{3. What happens if the price of \HAV{} changes?} \\

\noindent All issuance of \NOM{} is done at the current $P_h$. However, when $P_h$ changes, the quantity of escrowed \HAV{} changes with it (not the \emph{value}). An increase in $P_h$ means that fewer of Bob's \HAV{} are escrowed. By contrast, a decrease in the $P_h$ means that more of his \HAV{} are escrowed. This process occurs automatically in order to ensure that the system remains overcollateralised. \\ 

\noindent \emph{4. What happens if the price of \NOM{} changes?} \\ 

\noindent In order to release escrowed \HAV{}, Bob must return the same quantity of \NOM{} that he issued. This means that if $P_n$ has increased in the market, he will need to spend more ether than he received when he issued in order to release his \HAV{}. Conversely, if $P_n$ has decreased, Bob will need to spend less in order to release his \HAV{}.

\subsubsection{Havven Wallet}

\todo[inline]{add diagram of wallet balances.}

\newpage

\section{System Analysis}

\newpage

\subsection{Nomin Demand and Havven Value}

\noindent Being freely-tradable ERC20 tokens, havvens will have a market price which, like
the nomin price, can be measured with an oracle.
Initially, while nomin demand is low, we will use a seven day rolling average of the market price for both havvens and nomins.
This rolling average is designed to smooth out fluctuations in the market price and increase the cost of attacking the system.\\

\noindent However, once nomin transaction volume is sufficiently high, we may instead consider internally estimating
the value of a havven by the fees it is likely to accrue in the future. This value, which implicitly measures nomin volume,
would allow issuance incentives to directly reflect changes in nomin demand.
By using this value instead of the havven market price, the system can avoid the influence of speculation,
since the permitted nomin supply would expand and contract in line with how much nomins are actually being used. \\

\noindent While the system cannot perfectly determine future fee returns and hence nomin demand, it is possible to estimate as a
function of the transaction fees that the system has recently generated.
This computation is designed not to be vulnerable to instantaneous volume spikes, while taking the most recent transaction
volumes to be the most highly-correlated with future volumes:

\vspace{3mm}

\begin{equation}
    V_{t} \ = \ \sum_{t'=1}^{t} \frac{F_{t - t'}}{(1 + r)^{t'}} \label{eq:price}
\end{equation}

with
\begin{align*} 
V_{t} \ \ & \text{ the system's valuation of a havven in period } t  \\
F_t \ \ & \text{ the total fees collected in period } t\\
r \ \ & \text{ a falloff term}  \\
\end{align*}

\noindent This can be computed efficiently, because $V_{t+1} = \frac{V_t + F_t}{r}$. 
Further, if it is assumed that the average fee take is approximated by $F_t$, and $t$ is large, then:

\vspace{2mm}

\begin{equation}
    V_t \ \approx \ \sum_{t'=1}^{\infty} \frac{F_t}{(1 + r)^{t'}} \ = \ \frac{F_t}{H \cdot r}
\end{equation}

\vspace{3mm}

\noindent Consequently, $\frac{1}{r}$ approximates the number of periods for a havven to yield a fee return of $V_t$.
A judicious choice of $r$ can then yield a $V_t$ which underestimates the market price of havvens (which also incorporates,
for example, capital gains), while not unduly constraining nomin supply.

\newpage

\subsection{Fee Evasion}

\noindent Being based on Ethereum, Havven is potentially vulnerable to its
tokens being wrapped by another smart contract which takes deposits, and
replicates all exchange functionality on redeemable IOU tokens it issues.
These wrapped tokens could then be exchanged without incurring fees.
\noindent We consider this situation unlikely for a number of reasons. \\

\noindent First, the fees are designed to be low enough that most users
shouldn't notice them, so users will not in general be strongly motivated to
switch to a marginal and less trustworthy alternative. Additionally, transfer
fees will still be charged upon deposit into and withdrawal from
token-wrapping contracts, which partly constrains the utility of the wrapper.

\noindent Second, network effects are tremendously important for currencies;
in order to have utility a token must be accepted for exchange in the
marketplace. This is challenging enough in itself, but a wrapped token must
do this to the extent that it displaces its own perfect substitute: the token
it wraps.
In Havven's case, this would undermine its built-in
stabilisation mechanisms, which become more powerful with increased
transaction volume. Consequently, as a wrapped nomin parasitises more of the
nomin market, it destroys the basis of its own utility, which is that nomins
themselves are stable.

\noindent Finally, it is unlikely that a token wrapper will be credible, not
having been publicly and expensively audited, while its primary function
undermines the trustworthiness of its authors. \\

\noindent Even as token wrapping may appear unlikely to the authors, there
are at least two remedies which can be instituted to resolve this. \\

\noindent It would be a simple matter to implement a democratic remedy,
weighted by havven balance, by which havven holders can freeze or confiscate the
balance of any contract that wraps assets. Those havven holders are
incentivised not to abuse this system for the same reason that bitcoin
mining pools do not form cartels to double-spend: because abuse of this
power would undermine the value of the system, and thus devalue their
own holdings.

\noindent The credible threat of such a system existing is enough to
discourage token wrappers from being used, even if they are written, since
any user who does so risks losing their entire wrapped balance. \\

\noindent An alternative solution is to institute a hedging fee, charged on static nomin
balances. Such a fee could be discounted against transfer fees so that it
encourages general token velocity, and imposes a cost on those who buy nomins
simply in order to hedge, and are thus a risk for the network. Under this
model, a token wrapper would not dodge this fee: the full balance that was
wrapped will not be available at the time of unwrapping.

\include{tex/system_analysis_subsections/analytic_case_study}

\newpage

% \appendix
% \appendixpage
% \addappheadtotoc
%\input{tex/alternative_approaches}
%
\section{System variables}

% \todo[inline]{More complete system variable section.}

\noindent What follows are the main variables of the system. Under each
heading, each row will correspond to a single quantity of interest. Each row
will have three columns. Leftmost, a mathematical definition of the variable;
in the middle, the dimension of the quantity (which units it is measured in);
and on the rightmost, a short English summary of the variable. Items without
a dimension provided are dimensionless quantities. \\

\noindent Certain abbreviations will be used. For example, \(\HAV{}\) and
\(\NOM{}\) will be used as abbreviations for havvens and nomins considered as
units of measurement. \\

\paragraph{Prices and Money Supply}
\begin{align*}
    &P_h \ && &(\frac{\text{\$}}{\HAV{}}) && &\text{: havven price.} \\
    &P_n \ && &(\frac{\text{\$}}{\NOM{}}) && &\text{: nomin price.} \\
    \intertext{}
    &H \ && &(\HAV{}) && &\text{: Total supply of havvens; constant at \(10^8\) havvens.} \\
    &N \ && &(\NOM{}) && &\text{: Total supply of nomins; floating.} \\
    &H_i \ && &(\HAV{}) && &\text{: Quantity of havvens owned by account \(i\).} \\
    &N_i \ && &(\NOM{}) && &\text{: Quantity of nomins owned by account \(i\).} \\
\end{align*}

\paragraph{Collateralisation Ratios}
\begin{align*}
    &C &= \frac{P_n \cdot N}{P_h \cdot H} \ && & && &\text{: Global issuance ratio. } \\
    &C_{opt} &= \frac{P_n \cdot N}{P_h \cdot H} \ && & && &\text{: Global targeting issuance ratio. Changes in response to \(P_n\). } \\
    &N_{opt} &= C_{opt} \cdot P_h \cdot H \ && &(\NOM{}) && &\text{: Targeted nomin supply to encourage.} \\
    \intertext{Under most conditions, we expect \(0 \leq C_{opt} \leq C_{max} \leq 1\)}
\end{align*}
\\

\paragraph{Microeconomic Variables} These should be defined as functions of \(P_n, \ P_v, \ \text{fees, etc.}\)
\begin{align*}
S_n \ && (\frac{1}{\text{sec}}) && &\text{: average nomin spend rate} \\
S_i \ && (\frac{1}{\text{sec}}) && &\text{: average issuance rate} \\
S_r \ && (\frac{1}{\text{sec}}) && &\text{: average redemption rate}
\end{align*}
\\

\paragraph{Money Movement}
\begin{align*}
    V_n &= S_n \cdot N \ && &(\frac{\NOM{}}{\text{sec}}) && &\text{: nomin transfer rate.} \\
    V_v &= V_i + V_r \ && &(\frac{\HAV{}}{\text{sec}}) && &\text{: nomin} \leftrightarrow \text{havven conversion rate.} \\
    V_i &= (C - C_N) \cdot S_i \ && &(\frac{\HAV{}}{\text{sec}}) && &\text{: nomin issuance rate.} \\
    V_r &= C_N \cdot S_r \ && &(\frac{\HAV{}}{\text{sec}}) && &\text{: havven redemption rate.} \\
    \intertext{\(V_i\) is assumed to grow as there are more free havvens in the system.}
    \intertext{\(V_r\), by contrast, is taken to grow proportionally with the number of escrowed havvens.}
\end{align*}
\\

\paragraph{Fees}
\begin{align*}
\intertext{The following fees are ratios, for example 0.1\%, levied on each transaction.}
&F_{nx} & \ && &\text{(dimensionless)} && &\text{: nomin transfer fee} \\
&F_{cx} & \ && &\text{(dimensionless)} && &\text{: havven transfer fee} \\
&F_i & \ && &\text{(dimensionless)} && &\text{: nomin issuance fee} \\
&F_r & \ && &\text{(dimensionless)} && &\text{: havven redemption fee} \\
% \intertext{These quantities are the aggregated fees accrued by the system per unit time.}
% &Ag_{nx} &:= V_n \cdot F_{nx} \ && &(\frac{\NOM{}}{\text{sec}}) && &\text{: fees taken from nomin transfers.}
\end{align*}

\pagebreak


%\bibliography{tex/citations}
%\bibliographystyle{plain}

\end{document}
