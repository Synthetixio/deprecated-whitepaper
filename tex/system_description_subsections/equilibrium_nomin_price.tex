\subsection{Equilibrium Nomin Price}

\noindent We first introduce the core system variables:

\begin{align*}
H &\text{\ \ havven quantity} & N &\text{\ \ nomin quantity} \\
P_h &\text{\ \ havven price}  & P_n &\text{\ \ nomin price} \\
\end{align*}

% \todo[inline]{Consider renaming havven price symbol to reflect that it is computed from income}

\noindent All havven tokens are created at initialisation, so \(H\) is
constant. The quantity of nomins floats, responding to the issuance actions
of havven holders. The Havven system needs to incentivise issuers to maintain
\(N\) such that the nomin price \(P_n\), is stable at \$1. As we proceed, we may
subscript variables with \(t\) to indicate the value of that variable at a
given time. Any variable lacking such a subscript indicates the value of the
quantity it represents at the current time. \\

\noindent In Havven, the measure of the value of nomins against the value of
havvens is called the collateralisation ratio:

\begin{equation}
C \ = \ \frac{P_n \cdot N}{P_h \cdot H} \label{eq:collateralisation}
\end{equation}

\vspace{3 mm}

\noindent From the law of supply and demand, at a given level of demand,
there exists some supply of nomins \(N^*\) which yields an equilibrium
price of \$1. This quantity is associated with an ideal collateralisation
ratio at a particular havven price \(C^*_{P_h}\) which is achieved when \(N = N^*\).
Therefore, when \(C = C^*_{P_h}\), \(P_n = \$1\).
We visualise this equilibrium below with hypothetical demand and supply curves. \\

% \todo[inline]{curved lines}
% \todo[inline]{make diagrams pretty}
\begin{center}
\begin{tikzpicture}[scale=3]
    % draw axes
    \draw [<->, thick] (0,2) node (yaxis) [above] {\(P_n\)} |- (2.5,0) node (xaxis) [right] {\(N\)};
    % draw intersecting lines
    \draw (0.5, 0.5) coordinate (a_1) -- (2,1.8) coordinate (a_2) node[pos=0.0, left] {S};
    \draw (0.5, 1.8) coordinate (b_1) -- (2,0.5) coordinate (b_2) node[pos=1.0, right] {D};
    % calculate coordinate of intersection
    \coordinate (c) at (intersection of a_1--a_2 and b_1--b_2);
    \draw[dashed] (yaxis |- c) node [left] {\(\$1\)} -| (xaxis -| c) node[below] {\(N^* = C^*_{P_h} \cdot P_h \cdot H\)};
\end{tikzpicture}
\end{center}

% Alternate demand curve
%\begin{center}
%\begin{tikzpicture}[scale=3]
%    % draw axes
%    \draw [<->, thick] (0,2) node (yaxis) [above] {\(P_n\)} |- (2.5,0) node (xaxis) [right] {\(N\)};
%    % draw intersecting lines
%    \draw[blue, very thick] (0.5, 0.55) -- (1.33,1.13) -- (2, 1.13) -- (2.5, 1.63) node[pos=1.0, left] {S};
%    \draw[red, very thick] (0.5, 1.53) -- (1.0, 1.03) -- (1.83, 1.03) -- (2.5, 0.45) node[pos=1.0, right] {D};
%    % calculate coordinate of intersection
%    % \coordinate (c) at (intersection of a_1--a_2 and b_1--b_2);
%    % \draw[dashed] (yaxis |- c) node [left] {\(\$1\)} -| (xaxis -| c) node[below] {\(N_{opt} = C_{opt} \cdot P_h \cdot H\)};
%\end{tikzpicture}
%\end{center}

%(5, 5) (20, 18)
%y = 13/15x - 2/3

%(5, 18) (20, 5)
%y = -13/15x + 67/3

%intersection @
%(3*5*23/2*13, 5*13/2*3) = (345/26, 65/6) = (13.3, 10.8)
\noindent We assume that some level of demand exists given the utility of nomins as a stable
cryptocurrency. Although the organic supply and demand for nomins cannot be easily
manipulated, all issuance is controlled by havven holders, whose incentives are in turn
controlled by the system. It follows that as we require a fixed price \(P_n = \$1\)
and are unable to control either \(P_h\) or \(H\), we must incentivise havven holders
to manipulate \(N\), and therefore \(C\), in order to satisfy our requirement.
