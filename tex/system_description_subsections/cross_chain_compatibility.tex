\subsection{Cross Chain Compatibility}

\subsubsection{Capturing Transactions}

\noindent Havven has been designed to support multiple blockchains. The Ethereum
chain was chosen for the initial launch because of the maturity of the ecosystem.
Due to reliance on fees generated from transactions, we must be able to capture
transaction volume across multiple chains. This will allow havven to scale 
faster and de-risk scaling and other challenges on each individual chain. \\

\subsubsection{Deployment Strategies and Tradeoffs}

\noindent There are two approaches to deploying Havven to a new chain, direct
and indirect collateralisation. Direct collateralisation relies on establishing a
collateral pool on the new chain through minting Havven tokens on that chain. 
Indirect collateralisation does not require the establishment of a new collateral 
pool, instead an atomic swap contract is created to allow nomins to be transferred 
between the chains. This means transactions on the second chain will still accrue 
fees, but these fees will flow back to the original chain. Given that these chains
are competing for users, this may create issues for adoption on the second chain. 
The solution to this may be to deploy a native Havven blockchain, which would make 
it possible to migrate the collateral pools back a single chain and reduce 
interchain competition for fees. Given these tradeoffs the likely path will be 
direct collateralisation.\\

\subsubsection{Future Considerations}

\noindent Having distinct collateral pools will allow for different governance 
structures to be employed across each chain as well as experimentation with system 
variables.\\

\noindent It could also be through a swap between the two chains, in this case a 
holder of HAV on Ethereum would swap their HAV.ETH tokens for HAV.newchain. These 
two approaches each have tradeoffs. In the first approach the network will need to 
be bootstrapped again starting from zero volume. In the second approach the collateral 
value will be split between the two chains, this could have a short term impact on 
the circulating currency on that chain. One advantage of the second approach, though, 
is that it allows the market to move the collateral to whichever chain demands it 
based on the transaction fees generated on that chain.\\
