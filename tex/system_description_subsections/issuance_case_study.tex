\newpage

\subsection{Incentives Example} In this section we illustrate how Havven's inecentives encourage havven holders to change the supply of nomins when the price changes. We consider a simple example with two havven holders.

\subsubsection{Initial Conditions} The interest rate (R), transaction fee rate (k) and velocity (v) are:

\begin{align*}
R &= 10\% & k &= 3.33\% & v &= 6
\end{align*}

\noindent The values of the price sensitivivty parameter ($\sigma$), the flattening parameter ($\phi$) and the $C_{max}$ multiplier ($a$) are: 

\begin{align*}
\sigma &= 50 & \phi &= 3 & a&= 1.25
\end{align*}

\noindent The inverse demand function for nomins is:

\begin{gather} \label{eq:nomindemand}
\begin{align}
\begin{split}
P_n &= \frac{\varepsilon GDP}{vN}
\end{split}
\end{align}
\end{gather}

\noindent The initial value of $\varepsilon GDP = 900$. Havven holder $1$ and $2$ have the following havven and nomin issuance balances:

\begin{align*}
H_1 &= 100 & N_1 &= 50 \\
H_2 &= 200 & N_2 &= 100
\end{align*}

\begin{table}[!htbp]
	\centering
	\begin{tabular}{|m{1cm}|m{1cm}|m{1cm}|m{1cm}|m{1cm}|m{1cm}|m{1cm}|m{1cm}|}
		\hline
		\text{$P_{n,-1}$}&\text{$P_{h,-1}$}&\text{$C_{-1}$}&\text{$C_{1,-1}$}&\text{$C_{2,-1}$}&\text{$f(P_{n,-1})$}&\text{$C_{opt,-1}$}&\text{$C_{max,-1}$}\\
		\hline
		1 & 1 & 0.5 & 0.5 & 0.5 & 1 & 0.5 & 0.625\\
		\hline
	\end{tabular}
	\caption{Initial system conditions}
	\label{table:initial conditions}
\end{table}

\noindent From equation \eqref{eq:feesreceived}, the value of $\alpha_{r,-1} = 0.10$. We can express the expected profit, $\pi$, of each havven holder as follows: 

\begin{gather} \label{eq:nomindemand}
\begin{align}
\begin{split}
\pi_{1,-1} &=\alpha_{r,-1}\frac{H_1}{R}=0.10\cdot \frac{100}{0.1}=100,\\ \\
\pi_{2,-1} &=\alpha_{r,-1}\frac{H_2}{R}=0.10\cdot \frac{200}{0.1}=200.
\end{split}
\end{align}
\end{gather}

\noindent The number of escrowed havvens for each havven holder is $80$ and $160$ respectively, given by equation \eqref{eq:escrowed}. 

\newpage

\noindent At the beginning of period $t=0$, there is a 10\% decrease in $\varepsilon GDP$. Since the supply of nomins $N$ has not changed yet and the velocity $v$ is assumed to be fixed, the price $P_n$ decreases from \$1 to \$0.90.

\subsubsection{Neither havven holder changes}

\begin{table}[!htbp]
	\centering
	\begin{tabular}{|m{1cm}|m{1cm}|m{1cm}|m{1cm}|m{1cm}|m{1cm}|m{1cm}|m{1cm}|}
		\hline
		\text{$P_{n,0}$}&\text{$P_{h,0}$}&\text{$C_0$}&\text{$C_{1,0}$}&\text{$C_{2,0}$}&\text{$f(P_{n,0})$}&\text{$C_{opt,0}$}&\text{$C_{max,0}$}\\
		\hline
		0.9 & 0.9 & 0.5 & 0.5 & 0.5 & 0.905 &  0.4525 & 0.5656 \\
		\hline
	\end{tabular}
	\caption{Negative shock; no change; t = 1}
	\label{table:negative shock_both idle}
\end{table}

\noindent If both havven holders continue to keep their number of nomins constant, their expected profits from fees are the same as before:

\begin{align}
&\pi_{1,0}=\alpha_{r,i,0}\frac{H_{1}}{R}=\alpha_{base,0} \frac{C_{max,0}-C_{1,0}}{C_{max,0}-C_{opt,0}} \frac{H_{1}}{R}=100,\\
&\pi_{2,0}=200.
\end{align}

\noindent Even though they earn the same fees, their number of locked havvens which cannot be freely traded without buying back the issued nomins, has increased to $\check{H}_{1,0} = 88.4$ and $\check{H}_{2,0}=176.8$. 

\subsubsection{Both havven holders change} Each havven holder has an opportunity to increase the proportion of fees they receive, by changing their number of issued nomins $N_{i,1}$ to align with $C_{opt}$. Since the fee collection is maximized when $C_{opt,1} = C_{i,1}$, they choose:

\begin{gather} \label{eq:nominissued}
\begin{align}
\begin{split}
N_{i,1}=C_{opt,1}P_{h,1}H_i/P_{n,0}
\end{split}
\end{align}
\end{gather}

\begin{table}[!htbp]
	\centering
	\begin{tabular}{|m{1cm}|m{1cm}|m{1cm}|m{1cm}|m{1cm}|m{1.5cm}|m{1cm}|m{1cm}|}
		\hline
		\text{$P_{n,1}$}&\text{$P_{h,1}$}&\text{$C_1$}&\text{$C_{1,1}$}&\text{$C_{2,1}$}&\text{$f(P_{n,1})$}&\text{$C_{opt,1}$}&\text{$C_{max,1}$}\\
		\hline
		0.9945 & 0.9 & 0.500 & 0.500 & 0.500 & 0.999 & 0.499  & 0.625 \\
		\hline
	\end{tabular}
	\caption{Negative shock; both change; t = 1}
	\label{table:negative shock both follow mechanism}
\end{table}

\noindent After both holders change their nomins, $\alpha_{base,1}=0.091$. In this case, the expected profit of each holder is:

\begin{gather} \label{eq:nomindemand}
\begin{align}
\begin{split}
\pi_{1,1}=\alpha_{base,1} \frac{H_{1}}{R}+(N_{1,1}-N_{1,0})P_{n,0}=86.22, \\ \\
\pi_{2,1}=172.45,
\end{split}
\end{align}
\end{gather}

\noindent The number of locked havvens for each havven holder has also reverted back to the level before the original price change $\check{H}_{1,1}=80$ and $\check{H}_{2,1}=160$.The new nomin price $P_{n,1}$ is very close to $1$ and $f(P_{n,1})\approx 1$ and $C_{opt,1}\approx C_{i,1}$. Therefore, neither havven holder has any incentives to change again the supply of nomins $N$, since they are receiving the maximum possible fees. 

\subsubsection{Havven holder 1 changes} We now consider the scenario in which only havven holder $1$ decides to change their number of issued nomines. The value of $\alpha_{base,1}=0.12$ and therefore the expected profits for each havven holder are:

\begin{align}\label{pi_neg_shock_only N1_ t=1}
\left.\begin{array}{l}
\pi_{1,1}=\alpha_{base,1}\frac{C_{1,1}}{C_{opt,1}} \frac{H_{1}}{R}+(N_{1,1}-N_{1,0})P_{n,0}=111.79,\\ \\
\pi_{2,1}=\alpha_{base,1}\frac{C_{max,1}-C_{2,1}}{C_{max,1}-C_{opt,1}} \frac{H_{2}}{R}+(N_{2,1}-N_{2,0})P_{n,0}=174.43.
\end{array}\right.
\end{align}

\noindent $P_{n,1}$ is still lower than one, $f(P_{n,1})\neq 1$, and $C_{1,1}\neq C_{opt,1}$. Thus, havven holder $1$ still has incentives to change $N_1$ (recall we assumed that havven holder $2$ is idle). Havven holder $1$ will continue to change the number of nomins while he has incentives to do so. Next, we present the results of iterating through subsequent periods:

\begin{table}[!htbp]
	\centering
	\begin{tabular}{|m{1cm}|m{1cm}|m{1cm}|m{1cm}|m{1cm}|m{1.5cm}|m{1cm}|m{1cm}|}
		\hline
		\text{$P_{n,6}$}&\text{$P_{h,6}$}&\text{$C_6$}&\text{$C_{1,6}$}&\text{$C_{2,6}$}&\text{$f(P_{n,6})$}&\text{$C_{opt,6}$}&\text{$C_{max,6}$}\\
		\hline
		0.921 & 0.9 & 0.500 & 0.477 & 0.512 & 0.953 & 0.477  & 0.596 \\
		\hline
	\end{tabular}
\end{table}
\begin{table}[!htbp]
	\centering
	$\Rightarrow$\begin{tabular}{|m{1cm}|m{1cm}|m{1cm}|}
		\hline
		\text{$\alpha_{base,6}$}&\text{$\pi_{1,6}$}&\text{$\pi_{2,6}$}\\
		\hline
		0.122 & 118.53 & 171.58 \\
		\hline
	\end{tabular}
	\caption{Negative shock; only havven holder $1$ reacts; $t=6$}
	\label{table:negative shock only 1 reacts t=6}
\end{table}

\noindent Havven holder $1$ profit improves after each iteration at the expense of havven holder $2$'s profit. However, $P_{n}$ stabilizes around $\$0.921$ instead of $\$1$. The reason being that, although $P_n\neq 1$, $C_{1,6}$ is similar to $C_{opt,6}$. As a consequence, havven holder $1$ is already getting the highest possible amount of fees and has no incentives to change his number of nomins anymore.

\subsubsection{Havven holder 2 changes} Finally, we consider the case in which havven holder $1$ remains idle and havven holder $2$ changes the number of nomins following the proposed mechanism.

\begin{table}[!htbp]
	\centering
	\begin{tabular}{|m{1cm}|m{1cm}|m{1cm}|m{1cm}|m{1cm}|m{1cm}|m{1cm}|m{1cm}|m{1.5cm}|m{1cm}|m{1cm}|}
		\hline
		\text{$P_{n,6}$}&\text{$P_{h,6}$}&\text{$C_6$}&\text{$C_{1,6}$}&\text{$C_{2,6}$}&\text{$f(P_{n,6})$}&\text{$C_{opt,6}$}&\text{$C_{max,6}$}\\
		\hline
		0.939 & 0.9 & 0.500 & 0.522 & 0.489 & 0.978 & 0.489  & 0.612 \\
		\hline
	\end{tabular}
\end{table}
\begin{table}[!htbp]
	\centering
	$\Rightarrow$\begin{tabular}{|m{1cm}|m{1cm}|m{1cm}|}
		\hline
		\text{$\alpha_{base,6}$}&\text{$\pi_{1,6}$}&\text{$\pi_{2,6}$}\\
		\hline
		0.105 & 77.25 & 204.9 \\
		\hline
	\end{tabular}
	\caption{Negative shock; only havven holder $2$ reacts; $t=6$}
	\label{table:negative shock only 2 reacts t=6}
\end{table}

\noindent In this case, havven holder $2$ improves his profits (Table \ref{table:negative shock only 2 reacts t=6} vs. Table \ref{table:negative shock only 2 reacts t=1}) at expense of the other havven holder's profits. Again, $P_n$ does not stabilize at $1$ but does at a price closer to $1$ than in the previous case, since havven holder $2$ has more impact over the supply of nomins.

\subsubsection{Conclusions} Both havven holders would rather change their number of nomins. Although for each of them the best scenario would be if the rival does not do anything, this scenario cannot be an equilibrium. This can be seen from the following strategic game representation of the previous analysis (for this representation, we assume that all iterations are made instantaneously and simultaneously by both havven holders).

\begin{table}[!htbp]
	\centering
	\begin{tabular}{|c|c|c|}
		\hline
		\text{}&\text{$N_{2,0}$}&\text{$N_{2}^*$}\\
		\hline
		\text{$N_{1,0}$} & 100 , 200 & 77.25 , 204.9 \\
		\hline
		\text{$N_{1}^*$} & 118.53 , 171.58 & 86.21 , 172.43 \\
		\hline
	\end{tabular}
	\caption{Negative shock; strategic game representation}
	\label{table:negative shock_strateg game represent}
\end{table}

\noindent $N_{i,0}$ is the action of maintaining the same number of nomins taken by holder $i$. Conversely, $N_i^*$ is the action of changing the number of nomins. Each box has the payoff that both holders get by choosing some particular action. For example, if havven holder $1$ chooses $N_{1,0}$ and holder $2$ chooses $N_{2,0}$, the former gets a payoff of $100$ and the latter $200$. It can be checked that havven holder $1$ will choose $N_{1}^*$ no matter what action is chosen by havven holder $2$ ($1$ gets larger payoffs following $N_{1}^*$ for any action that $2$ can take). Similarly, $2$ will choose $N_{2}^*$ no matter what the action of havven holder $1$ is. In other words, action $N_i^*$ strictly dominates remaining idle with $N_{i,0}$. Therefore, $\{N_1^*,N_2^*\}$ is the unique Nash equilibrium. \\

\noindent Moreover, this equilibrium yields a stable nomins price $P_n=\$1$ in accordance with the project's claim.
