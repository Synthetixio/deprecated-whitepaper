
\section{Qualitative Scenario Analysis}

This section provides a qualitative treatment of the reaction of the Havven system in response to various scenarios listed below:

%\todo[inline]{Scenario analysis.}

\subsection{Scenarios}

\paragraph{Havvens appreciate against nomins.}
\begin{itemize}
	\item{Havven-holders can issue more nomins.}

	This might be scary because we either want the nomin supply to fall, or nomin demand to increase.
	This is fine if the nomin price fell, because then nomin demand should increase.
	However, if it was simply that the havven price increased, then this might encourage
	oversupply of nomins as issuers compete for shares of the fee pool.

	\item{Cheaper exit}

	Anyone who has previously issued havvens but who wants to exit can buy nomins for
	cheaper than they issued them at, so egress with a profit.
	Alternatively, if the havven price doubles, then only half of their stake is required
	to back the nomins they have issued. They can use these proceeds to completely liquidate
	their position.

	\item{Each nomin locks fewer havvens.}
	
\end{itemize}

\paragraph{Havvens depreciate against nomins.}
\begin{itemize}
	\item{Havven-holders can issue fewer nomins.}
	
	Perhaps they may even be under-staked.

	\item{Each nomin locks more havvens.}
	\item{A player can issue a quantity of nomins and sell them for havvens.}
		  
	They would do this on the assumption that, once the nomin price decreases as a result of the increased
	supply, they will be able to buy back the same quantity of nomins to free up their
	havvens more cheaply.
\end{itemize}

\paragraph{Why would anyone issue nomins?}
\begin{itemize}
	\item Fees
	\item Because they thought the peg would break in the positive direction.
\end{itemize}

\begin{enumerate}
	\item Ratio moves favourably
	\item Ratio moves unfavourably
		\subitem Accumulate Havvens
			\subsubitem Few Havvens available.
			\subsubitem No Havvens available.
		\subitem Accumulate Nomis
			\subsubitem Few Nomins available.
			\subsubitem No Nomins available.
	\item Creation of new Havvens with new funds (not currently explored).
\end{enumerate}

\subsection{Expected Market Players}
%\todo[inline]{List expected players in the market.}
%\todo[inline]{Outline incentives and actions for different players.}

\begin{itemize}
	\item{Havven Holders}

	A havven-holder provides collateral and liquidity. It’s assumed havven-holders seek fee revenues,
	escrowing as many havvens as they can. This incentive only really makes sense if havvens are not
	significantly volatile over the long run. But in an unstable regime we also provide incentives
	for stabilising the nomin price. 

	\item{Nomin Users}

	Merchants, consumers, service providers, and so on: people who use the stable coin as a
	medium of exchange. They provide a base demand for nomins, which is necessary for fees
	to exist. These users may be disincentivised from using the system by excessive volatility
	in the price of nomins, or by high fees.

	\item{Arbitrageurs, Market Makers}

	The arbitrage force allows us to assume that the hav/nom, hav/fiat, nom/fiat
	prices are properly in alignment or will soon become aligned. Market making activities
	allow us to neglect the bid/ask spread, and situations where there is insufficient
	liquidity for players to transact.
	
	\item{Speculators}
	
	May tend to magnify price corrections, and are a significant vector by which to introduce
	exogenous shocks to the system modelling, e.g. large capital flows in response to breaking news.
	
	\item{Malicious Attackers}
	
	We should examine what happens if a George Soros (or otherwise) attacks Havven.
	%\todo[inline]{List the various possible attacks against the system.}

	\item{Central Banker}
	
	The Havven foundation will have significant capital reserves with which it could intervene
	in the market if necessary to stabilise nomin prices. The system should work without
	such actions, but in extreme situations it might be necessary to undertake them.
	The advantage of such a market participant is, given that a very large market entity is
	willing to underwrite the stability of the coin, profit strategies predicated upon the
	stability of the token become less risky, so more feasible. So the Havven foundation in
	this capacity takes on the role of providing confidence.

\end{itemize}



\pagebreak