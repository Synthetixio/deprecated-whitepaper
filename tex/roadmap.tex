\section{Road Map}

The final Havven system will be released in phases. This appendix describes
each of these phases and their expected completion dates.

\paragraph{Ether-Backed Nomins} (\texttt{eUSD})

Q2 2018

Description:
The ether-backed nomin system implements \texttt{eUSD}, an interim stablecoin.
\texttt{eUSD} operates while the system is being developed.
The \texttt{eUSD} price's stability is maintained by a pool of ether which
backs the circulating stablecoin supply. In this system, \texttt{eUSD} can be
purchased from, and sold into, the pool for \(US\$1\) worth of ether.
The Havven foundation provides at least \(US\$2\) of ether collateral for each
issued \texttt{eUSD}, which ensures that the entire supply is redeemable for its
face value even in the face of the ether price falling by up to two thirds.
Fees are collected on \texttt{eUSD} transactions, and these fees are collected
by havven owners.

The \texttt{eUSD} system, being based on ether collateral, has a constrained maximum supply.
Therefore, in order to scale, the system must move towards an issuance mechanism based on havven-backed
nomins. System A is the first step towards the final such system.

Features:
\begin{itemize}
    \item{\texttt{eUSD}}
    \item{havvens}
    \item{An ether price oracle}
\end{itemize}


\paragraph{System A} \texttt{nUSD}: Static C, proportional fees, and white-listed issuers

Q2 2018

Description:
The System A version of \texttt{nUSD} allows issuance up to a static
collateralisation ratio against havvens, which is set by the foundation.
\texttt{nUSD} will be issued directly into the issuer's wallet, and fees will be paid
proportionally with the number of issued nomins per user. Given that it's necessary to
encourage liquidity, but not all the mechanisms outlined in this paper will be operating
yet, issuance will be by the foundation itself, and potentially other white-listed
addresses it trusts. In this way, the stability of the token is maintained by direct
market intervention by the Havven foundation.

As System A fundamentally changes the issuance mechanics, \texttt{nUSD} is a distinct
token from \texttt{eUSD}, with \texttt{eUSD} exchangeable one-for-one with \texttt{nUSD}
through the Havven foundation at the time of the \texttt{nUSD} launch, or exchangeable
for \(US\$1\) worth of ether, as usual. After a liquidation period, the \texttt{eUSD} contract
will be destroyed.
Future updates to the Havven system will not entail the destruction of any nomin tokens; thus
both \texttt{havvens} and \texttt{nUSD} will persist without further interruption.

System A limits issuance and fee returns to the Havven foundation itself. Therefore
only those havvens the foundation controls can be used to issue nomins against, which means
the full value of the havven network cannot be deployed for issuance. The fact that nomins
are created directly in the issuer's wallet may limit liquidity if issuers choose not to sell
those nomins. Systems B and C aim to combat these limitations.

Features:
\begin{itemize}
    \item{\texttt{nUSD}}
    \item{Havven-backed issuance mechanics}
    \item{A havven price oracle}
\end{itemize}


\paragraph{System B} \texttt{nUSD}: Universal non-discretionary issuance

Q3 2018

Description:

System B will enforce that issuance occurs through an issuance controller or
decentralised exchange to ensure that new liquidity is actually injected
into the market. The Havven system will have to manage the volume of
issue/burn orders on either side of the book while also continually updating
the price at which they are offered.


\paragraph{System C} \texttt{nUSD}: Dynamic fee computations

Q4 2018

Description:

The full featured incentive mechanisms will be activated so that users earn fees
in accordance with how effectively they stabilise the nomin price.

Features:


\paragraph{System D} Multi-currency nomins

Q1 2019

Description:

Features:
