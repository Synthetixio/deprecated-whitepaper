\section{System Description} Havven is a dual-token system that, combined with a set of novel incentive mechanisms, stabilises the price of the nomin with respect to an external asset. \\

\noindent The havven token serves two functions:

\begin{itemize}
\item{To provide the system with collateral.}
\item{To allow actors to participate profitably in stabilising the nomin price.}
\end{itemize}

\paragraph{Collateralisation}

\noindent Confidence in stability of the nomin begins with overcollateralisation, so that the value of escrowed havvens is greater than the value of nomins in circulation. The value of havvens is derived internally by the system as a function of the demand for nomins; this decouples the value of the collateral pool from market speculation. \\

\noindent As long as the ratio of total nomin value to total havven value remains favourable, there is sufficient backing in the underlying collateral pool to ensure that nomins can be redeemed for their face value. The redeemability of a nomin for the havvens against which it was issued strongly supports a stable price.  

\paragraph{Incentives}

\noindent Havven rewards those that have issued nomins. These rewards are derived from transaction fees and are distributed in proportion with how well each issuer maintains the correct nomin supply. The system monitors the nomin price, and responds by adjusting its targeted global supply, which individual issuers are incentivised to move towards.  \\

\noindent In the case of extreme volatility stronger stabilisation mechanisms may be applied. For example extra fees, interest rates, or automated collateral recovery.

\newpage

\subsection{Definitions}

\noindent We first introduce the core system variables:

\begin{align*}
H &:= \text{havven quantity,} & N &:= \text{nomin quantity,} \\
P_h &:= \text{havven price,}  & P_n &:= \text{nomin price.} \\
\end{align*}

\noindent All havven tokens are created in the initial system state, so $H$ is constant. The quantity of nomins, $N$, floats in response to the actions of havven holders.The Havven system needs to incentivise havven holders to maintain $N$ such that the nomin price, $P_n$, is stable at \$1.\\

\noindent In Havven, the collateralisation ratio measures the value of nomins against the value of havvens:

$$ C = \frac{P_n * N}{P_h * H}. $$

\subsection{Nomin Equilibrium Price} The law of supply and demand states that there exists some supply of nomins, $N_{opt}$, where the related level of demand yields an equilibrium price of \$1. This quantity is associated with an optimal collaterisation ratio, $C_{opt}$. We visualise this situation below. \\

\todo[inline]{curved lines}
\todo[inline]{make diagrams pretty}
\begin{center}
\begin{tikzpicture}[scale=3]

% draw axes
\draw [<->, thick] (0,2) node (yaxis) [above] {$P_n$} |- (2.5,0) node (xaxis) [right] {$N$};

% draw intersecting lines
\draw (0.5, 0.5) coordinate (a_1) -- (2,1.8) coordinate (a_2);
\draw (0.5, 1.8) coordinate (b_1) -- (2,0.5) coordinate (b_2);

% calculate coordinate of intersection
\coordinate (c) at (intersection of a_1--a_2 and b_1--b_2);

\draw[dashed] (yaxis |- c) node [left] {$1$} -| (xaxis -| c) node[below] {$N_{opt} = \frac{C_{opt} * P_h * H}{P_n}$};

\end{tikzpicture}
\end{center}

\noindent The system is unable to influence the demand for nomins. We assume that some level of demand exists given the utility of nomins as a stable cryptocurrency. Although demand cannot be manipulated, the supply of nomins is controlled by havven holders, whose issuance incentives are in turn controlled by the system. It follows that as we require a fixed price $P_n = \$1 $ and are unable to control either $P_h$ or $H$, we must manipulate $C_{opt}$ such that $N = N_{opt}$ in order to satisfy our requirement.

\subsection{Issuance and Collateralisation} 

\noindent Havven's goal is to remain overcollateralised. In order to do so, the system defines a collaterisation target:

$$ 0 < C_{opt} < 1$$

\noindent It is necessary at this point to distinguish between the nomins in a wallet $N_i$ (equity) and the nomins that have been issued by that wallet $\check{N_i}$ (debt). Note that globally, the $\sum_{i}N_i = \sum_{i}\check{N_i}$, as all circulating nomins were issued by some wallet. However, a given wallet may have a balance different from its issuance debt.\\
\todo[inline]{Make sure this equity/debt language is not too security-like.}

\noindent Hence we can define the collateralisation ratio for an individual wallet $i$ in terms of its issuance debt:

$$ C_i = \frac{P_n \cdot \check{N_i}}{P_h \cdot H_i}$$

\noindent The system provides incentives for individual issuers to bring their $C_i$ closer to $C_{opt}$ while maintaining $C_{opt}$ itself at a level that stabilises the price.

\paragraph{Nomin Issuance}

\noindent The nomin issuance mechanism allows Havven to reach its collaterisation target. Issuing nomins escrows some quantity of havvens, which cannot be moved until they are unescrowed. The quantity of havvens $\check{H_i}$ locked in generating $\check{N_i}$ nomins is:

$$ \check{H_i} = \frac{P_n \cdot \check{N_i} \cdot C_{opt}}{P_h }$$

\noindent Under equilibrium conditions, this implies that $C_i = C_{opt}$ is attained for a wallet exactly when all havvens have been escrowed, at $H_i = \check{H_i}$. As a result, the issuer cannot take actions that would bring $C_i$ above $C_{opt}$ and undercollateralise their position. Instead, $C_i$ can only exceed $C_{opt}$ with price fluctuations. \\

\noindent After generating the nomins, the system places a \textbf{limit sell} order with a price of \$1 on a decentralised exchange. This means that the nomins will be sold at the current market price, down to a minimum price of \$1 USD. If we assume implementation on Ethereum, then the nomins are sold for ETH, with the proceeds of the sale remitted to the issuer. \\

\paragraph{Nomins Destruction}

\noindent In order to access the original havvens that have been escrowed, the issuer must return the same quantity of nomins to the system to be burned. Nomins can be purchased in the open market.

\subsubsection{Issuance Example (Diagrams to come?)}

\begin{enumerate}
\item{Bob purchases 10 havvens at \$10 each, total value \$100.}
\item{Bob decides to issue some nomins. He escrows 5 of his havvens worth \$50, locking them in the system.}
\item{The optimum collaterilisation ratio is 0.2. The system generates and sells 10 nomins on the market and transfers the proceeds, in ETH, to Bob's wallet.}
\item{In order to release the escrowed havvens, Bob must return 10 nomins.}
\end{enumerate}

\subsubsection{Price changes (Table here maybe?)}

\noindent \emph{What happens if the price of havvens changes?} \\

\noindent All issuance of nomins is done at the current $P_h$. However, when $P_h$ changes, the quantity of escrowed havvens changes with it (not the \emph{value}). An increase in $P_h$ means that fewer of Bob's havvens are escrowed. In contrast, a decrease in the $P_h$ means that more of his havvens are escrowed. This process occurs automatically in order to ensure that the system remains overcollateralised. \\ 

\noindent \emph{What happens if the price of nomins changes?} \\ 

\noindent In order to release escrowed havvens, Bob must return the same quantity of nomins that he issued. This means that if $P_n$ has increased in the market, he will need to spend more ether than he received when he issued in order to release his havvens. Conversely, if $P_n$ has decreased, Bob will need to spend less in order to release his havvens.

%\subsubsection{Global collaterilisation Ratio}

%\noindent Havven can ensure that nomins are overcollateralised at the time of issuance, however, a reduction in $P_h$ lowers the value of the escrowed havvens. The global collaterilisation ratio is defined as the total value of nomins divided by the total value of havvens:

%$$ U = \frac{P_n * N}{P_h * H}.$$ \\

%\noindent Intuitively, if $U = 1$, the value of nomins and havvens are equal. Hence, given our overcollateralisation property, the system targets $U <  1$. To do this, Havven only allows the issuance of nomins up to a maximum collaterilisation ratio.

%\subsubsection{Unique Collateralisation Ratio}

% \noindent Given the need to adjust the supply of nomins, an optimum collaterilisation ratio is defined as the point at which maximum incentives are applied:

% $$ 0 \leq U_{opt} \leq U_{max} \leq 1.$$

% \noindent Each havven holder that has issued nomins has a unique collaterilisation ratio, $ U_i $. The system can measure the degree to which their $ U_i $ is above or below the optimum and reward them accordingly. In this way the system incentivises the issuance and destruction of nomins. $ U_{opt} $ is defined formally below in terms of $ P_n $. As  the nomin price diverges from the desired \$1, the incentive to either expand or contract nomin supply increases.

\newpage
\subsection{Transaction Fees} In addition to overcollateralisation, the system needs a direct incentive mechanism that can react to changes in demand for nomins.

\subsubsection{Nomin transaction fees} Every time a nomin transaction occurs, the Havven system charges a small transaction fee. Transaction fees allow the system to generate revenue, which it can distribute to havven holders as an incentive to maintain nomin supply at $C_{opt}$. \\

\noindent The fee rate charged on nomin transactions is $\alpha_c$. It is constant and will be sufficiently small that it provides little to no friction for the user.\\

$$ a_c = k.$$ 

\begin{center}
\begin{tikzpicture}[scale=3]

% draw axes
\draw [<->, thick] (0,2) node (yaxis) [above] {$\alpha_c$} |- (2.5,0) node (xaxis) [right] {$P_N$};

% draw one line
\draw (0.0, 0.75) coordinate (a_1) -- (2.0,0.75) coordinate (a_2);

\coordinate (c) at (1.75, 0.75);

\draw[dashed] (yaxis |- c) node [left] {k};

\end{tikzpicture}
\end{center}

\newpage
\subsubsection{Fees received by Havven Holders}

\noindent The fee rate paid to a havven holder that has escrowed is $\alpha_r$. This rate changes with respect the individual's unique collaterilisation ratio, $C_i$. It increases linearly to a maximum at the optimal collaterilisation ratio $C_{opt}$, before quickly diminishing as $C_i$ approaches $C_{max}$. Beyond the maximum collaterilisation ratio $\alpha_r$ is 0. Note, $\alpha_r$ is applied to the pool of collected fees which is determined by $\alpha_c$. \\

$$ \alpha_{R,t,i} = \alpha_{base,t} * f_{i,t}(C_{i,t}, C_{opt}, C_{max,t}), $$

\[
f_{i,t}(C_{i,t}, C_{opt}, C_{max,t}) = 
\begin{cases}
 \frac{C_{i,t}}{C_{opt,t}} &\mbox{when } C_i \leq C_{opt}, \\[1em]
 \frac{C_{max,t} - C_{i,t}}{C_{max,t} - C_{opt,t}} &\mbox{when } C_{opt} \leq C_i \leq C_{max}, \\[1em]
 0 &\mbox{otherwise}.
 \end{cases}
\]

\begin{center}
\begin{tikzpicture}[scale=3]

% draw axes
\draw [<->, thick] (0,2) node (yaxis) [above] {$\alpha_r$} |- (2.5,0) node (xaxis) [right] {$C_i$};

% draw two lines
\draw (0.0, 0.0) coordinate (a_1) -- (1.75,0.75) coordinate (a_2);
\draw (1.75, 0.75) coordinate (a_1) -- (2.25,0.0) coordinate (a_2);

\coordinate (c) at (1.75, 0.75);

\draw[dashed] (yaxis |- c) node [left] {$a_{base}$} -| (xaxis -| c) node[below] {$C_{opt}$};

\end{tikzpicture}
\end{center}

\noindent This fee distribution curve encourages havven holders who have escrowed to maintain their $U_i$ at $U_{opt}$.  \\

\newpage

\subsubsection{Deriving the base fee rate} The total amount of fees collected from users has to be equal to the total amount of fees paid to the havven holders. We define the total amount of fees collected, $A_{c,t}$ below: \\
$$ A_{c,t}  = \alpha_{c,t} * v_{n,t} * \sum\limits_I N_{I,t}, $$
$$ v_{n,t} \text{ the velocity of nomins at t}, $$
$$  \sum\limits_I N_{I,t} \text{ the total issued nomins}. $$
Next we define the total fees paid to havven holders $A_{R,t}$: \\
$$ A_{R,t} = \sum\limits_I \alpha_{R,t,I}. $$
Havven requires that: \\
$$ A_{R,T} =  A_{c,t}. $$ \\
Substituting our earlier definition of $\alpha_{R,t,I}$: \\
$$ \alpha_{base,t} *\sum\limits_I f_{i,t}(C_{i,t}, C_{opt}, C_{max,t}) =  \alpha_{c,t} * v_{n,t} * \sum\limits_I N_{I,t}. $$
Solving for $\alpha_{base,t}$:\\
$$ \alpha_{base,t} = \frac{\alpha_{c,t} * v_{n,t} * \sum\limits_I N_{I,t}}{\sum\limits_I f_{i,t}(C_{i,t}, C_{opt}, C_{max,t})}.$$

\noindent We have now defined the maximum fee rate, $\alpha_{base}$, in terms of the fees collected, $A_{c,t}$. This rate should be achieved when an individuals $U_i$ is at $U_{opt}$. \\

\noindent The definition of $C_{opt}$ must therefore provide the following incentive. If $P_n > 1$ then the system must encourage more nomin to be issued. If $P_n < 1$, the system must encourage nomin to be burned. 

\newpage

\subsection{Collaterilisation Ratio}
\subsubsection{Optimal collaterilisation Ratio}


\noindent The optimal collaterilisation ratio $C_{opt}$ is a target for havven holders to reach in order to maximise the amount of fees they receive. $C_{opt}$ is defined in terms of $P_n$ such that havven holders can influence the price of nomin through directly controlling the supply of nomin (a havven holder can change their individual collaterilisation ratio by buying or issuing more nomins). \\
 
\noindent The function for $C_{opt}$ given below provides our dynamic target for havven holders based on the price of nomin. The curve shows that the when $P_n$ is close to \$1, $ f'(P_n) $ is small. However, the further $P_n$ diverges from \$1, the larger the derivative becomes, providing an increasing incentive (via fees) for a havven holder to move toward $C_{opt}$.

$$ C_{opt} = f(P_n) * C,$$
$$ f(P_N) = max(\sigma * (x - 1)^{\phi} + 1, 0), $$
$$\text{where } 0 \leq \sigma, \text{ the price sensitivity parameter}, $$
$$\phi \geq 1, \text{ the flattening parameter}. $$ \\

\begin{center}
\begin{tikzpicture}[scale=1.15]
\begin{axis}[
    axis lines = left,
    xlabel = $P_N$,
    ylabel = {$f(P_N)$},
    xtick = {1},
    ytick = {0},
    y label style={at={(axis description cs:.15,1.00)}, rotate=270,anchor=south},
    x label style={at={(axis description cs:1.05,.13)},anchor=north},
]
\addplot[
domain=0:2,
range=0:2
]
{(x-1)^3 + 1};
\addplot [
domain=0:2,
range=0:2,
dashed
]  
 {x};
 \addplot[dashed, samples=50, smooth,domain=0:6] coordinates {(1,1)(1,0)};
\end{axis}
\end{tikzpicture}
\end{center}

\newpage

\subsubsection{Maximum collaterilisation Ratio}

\noindent Havven seeks to maintain $C \leq C_{opt} < C_{max} < 1$, in order to remain sufficiently overcollateralised. It might seem intuitive that $C_{max}$ should be a static value. However, since $C_{opt}$ changes linearly with $P_n$ and inversely with $P_h$, there are several situations where $C_{max}$ may need to change. Below we define $C_{max}$. \\

\[
U_{max} = 
\begin{cases}
 C_{base} &\mbox{when } C_{opt} \leq C_{base}, \\[1em]
 a * C_{opt} &\mbox{otherwise}.
 \end{cases}
\]

\begin{center}
\begin{tikzpicture}[scale=3]

% draw axes
\draw [<->, thick] (0,2) node (yaxis) [above] {$C_{max}$} |- (2.5,0) node (xaxis) [right] {$C_{opt}$};

% draw intersecting lines
\draw (0.0, 0.75) coordinate (a_1) -- (0.75,0.75) coordinate (a_2);
\draw (0.75, 0.75) coordinate (a_1) -- (1.8,1.8) coordinate (a_2);

% calculate coordinate of intersection
\coordinate (c) at (0.75,0.75);

\draw[dashed] (yaxis |- c) node [left] {$C_{base}$} -| (xaxis -| c) node[below] {$C_{base}$};

\end{tikzpicture}
\end{center}

\newpage

\subsection{Intrinsic Havven Price} With the havven token being ERC20 compliant, it will have a market price on both decentralised and centralised exchanges. \\

\noindent While the Havven system will access the current market price via a price oracle, it is beneficial to define a $P_h$ that can be determined internally to avoiding the influence of speculation. Ignoring speculative demand, $P_h$ can be expressed as a function of the transaction fees that the system charges. Below we define an initial iteration of the intrinsic $P_h$.

\begin{align*} 
P_{h,t} &= \frac{1}{H}* \sum\limits_{t=1}^\infty \frac{d_{n,t} *v_{n,t} * \alpha_{R,t}}{(1+R)^t} \approx \frac{d_{n,t} *v_{n,t} * \alpha_{R,t}}{R * H}, \\
& P_{h,t} \text{ is the price of one havven at time } t, \\
& H \text{ is the number of havvens}, \\
& d_{n,t} \text{ is the demand for nomins at t}, \\
& v_{n,t} \text{ is the velocity of nomins at t}, \\
& \alpha_{R,t} \text{ is the fee from trade with nomins}, \\
& R \text{ is the interest rate / rate of return of havvens}. \\
\end{align*}

\newpage